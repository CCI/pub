\section{The CCI Interface}

\subsection{Endpoints}
An endpoint is a virtualized instance of a device, it is the logical source 
or destination of all communications in CCI. An endpoint contains both a send 
and receive queue and their associated buffers, it is a complete container 
of all resources used by an application process or thread to perform CCI 
operations. As such, endpoints naturally fit the NUMA architecture, they can 
be bound to particular cores to maximize memory locality.

Endpoints interact with the application through events. CCI provides a 
function to immediately return the next event for a particular endpoint. 
Optionally, this function can block if no events are available, allowing the 
application thread to be scheduled out by the operating system. To facilitate 
the integration of the blocking semantic with non-CCI operations, each endpoint 
exposes an OS-specific handle such as file descriptor in Linux or an object in 
Windows.

The concept of endpoint is key to scalability. By multiplexing incoming 
messages into a shared receive queue and similarly buffering outgoing 
messages in a shared send queue, the overall memory footprint is independent 
of the number of peers communicating with the endpoint. On the time dimension, 
an endpoint offers a unified completion queue for events, allowing for 
OS-bypass implementations to provide low latency at scale.

\subsection{Connections}
CCI uses connections to represent the state of communication channels with 
remote peers. An endpoint can be connected to another endpoint through one or 
more connections. Connections have different attributes, such as 
reliability and order. CCI supports five different connection types:

\begin{itemize}
\item Reliable with Ordered completion (RO)
\item Reliable with Unordered completion (RU)
\item Unreliable with Unordered completion (UU)
\item Multicast Send (MC\_TX)
\item Multicast Receive (MC\_RX)
\end{itemize}

As previously stated, unreliable connections are useful for some applications. 
Order however is an unusual characteristic, as most network technologies such 
as Ethernet and InfiniBand assume order on the wire. 
Nevertheless, there are several situations when unordered semantics are 
desirable. For example, using multiple network links to increase bandwidth, 
technique also known as \emph{channel bonding}, naturally does not provide 
global packet ordering. Similarly, switch contention avoidance techniques, 
such as adaptive routing, do break order when different routes are selected 
between two endpoints. Switch contention is a major scalability concern, 
relaxing order at the communication interface level is believed to be 
essential to enable effective technological solutions.

Connections are established through a client-server process similar to 
Sockets or RDMA-CM\note{needs ref}. However, applications such as 
Apache\note{needs ref} commonly use a 2-step mechanism where clients 
initially connect to a broker thread, which passes the requests to a 
different thread for processing. This allow the application to 
transparently handle requests with multiple processing threads.
CCI goes a step further and provides a connection manager framework to 
the application. Clients initiate a connection by specifying a server 
Uniform Resource Identifier, a string that contains information used 
by the connection manager to identify the requested service. In the context of 
HPC, it could be a batch queue system job identifier and a rank. For web 
services, the URI could be a standard URL. The flexibility of the URI allows 
for extensibility of naming and support of additional functionality, such as 
system-wide load-balancing, fail-over, etc.

On the server side, the application binds to a service and receive 
connection requests independently of any endpoint. The incoming connection 
request carry a payload that can be used for identification or authentication. 
Upon accepting the request, an endpoint is selected to completed the 
connection. This indirection allows the application to choose a local endpoint 
at the last step of the connection process, effectively managing multiple 
endpoints on multi-core systems.

\subsection{Active Messages}
Buffering is the most efficient way to provide asynchronous semantics, which 
is essential for scalability. Sockets exclusively relies on buffering on both 
send and receive sides, a Sockets sends return as soon as the data has been 
written to the socket send buffer. Similarly, incoming payload is first 
written to the socket receive buffer and then retrieved by the application.

The Verbs interface provides asynchronous semantics through its send/recv 
operations. However, it delegates the buffer allocation to the application. 
Receive buffers have to be posted on a Queue Pair prior to messages arrival 
and they are matched in order, so they all have to be large enough for the 
biggest possible messages when using a Shared Receive Queue (SRQ). 
This put a practical limit to the maximum size of messages exchanged with 
this mechanism.

MPI offers explicitly buffered send but it is rarely used. Instead, the 
standard send may or may not buffer the message, it returns when 
the application buffer can be reused. In most implementation, it 
depends on the message size as explained in section~\ref{sec:mpi}. 
Small messages are expected, and sometimes assumed, to be buffered. Larger 
messages block the send as long as the message is not delivered to the receive 
side. Unfortunately, this assumed threshold is not defined by the MPI 
specification, it often results in unsafe code that can deadlock, breaking 
portability. 
The matching interface in MPI is very powerful but very complex. Support 
for wild cards require a single, coherent matching stack which cannot be 
offloaded to effectively handle intra-node messages. In addition, matching 
interfaces are stateful, which is greatly complicates fault recovery.

CCI uses a variations of Active Messages~\cite{voneicken-isca92} (AM) 
to address these issues. 
Most of AM's complexity, shared by GasNet\cite{gasnet}, is related 
to the use of asynchronous handlers. CCI uses events instead to greatly 
simplify the API. As the original AM implementation, buffers are managed 
internally from a constant pool, on both send and receive sides. 
On the receive side, the application is given access to a CCI buffer through 
a receive event. It may process the data in place or copy it out. When the 
application returns the receive event to the CCI library, the corresponding 
receive buffer is recycled. Receive events can be returned out of order, so 
the application can keep some receive buffers for some time if necessary. 
On the send side, messages are immediately buffered.

As with Verbs, the receive buffers have to be large enough for the largest 
message. As the buffer management is removed from the application, CCI 
explicitly defines a Maximum Message Size (MMS), specified by the underlying  
device. This well-defined limit avoids the portability and unsafe assumption 
of the MPI model. However, this maximum message size may be different for 
different devices, therefore for different endpoints. When two endpoints 
connect, the smallest MSS is used for that particular connection.

This model is particularly efficient when the MMS aligns with the Maximum 
Transfer Unit (MTU) of the underlying device, typically 2K to 8K. 
In this case, no inter-packet state is needed, which greatly simplify 
networking hardware requirements and allow CCI to leverage less 
sophisticated devices such as commodity Ethernet adapters. 
Some HPC networks use a very small MTU, such as 256 Bytes for 
IBM's BluGene\note{need ref}. Such networks most likely 
provide in-order hardware-based segmentation and reassembly, so they 
can use a larger maximum message size without performance penalty.

If the application desires to send a message larger than the limit, it has 
to perform segmention and reassembly itself. With a typical MSS in the order 
of a page size, the segmentation can effectively pipeline the copy into the 
send buffers with the packet injection itself. On the receive side, 
reassembly will require an extra memory copy, unless the application can 
manipulate the data in place over several non-contiguous receive buffers.
However, the segmentation/reassembly effor is a clear incentive for the 
application to restrict the Messages semantic in CCI to smaller traffic 
such as synchronization messages.

\subsection{Remote Memory Access}
For larger data movement, CCI provides a Remote Memory Access (RMA) semantic 
that enables zero-copy for low CPU overhead. RMA operations are one-sided, 
they are only allowed on reliable connections. 

The application needs to explicitely register memory to be used for RMA 
transfers. The memory registration process is specific to each device 
implementation but it usually consists in pinning the underlying physical 
pages and make them suitable for DMA operations. In CCI, memory is registered 
for a particular connection or for all connections. This allow for simpler 
remote memory protection than the Protection Domains of the Verbs API. A 
memory area can be registed on multiple CCI connections if needed.

By requiring explicit memory registration, CCI avoids the mistake of MPI 
where registration is not part of the API. It is performed in the critical 
path, which lead to various unsafe caching mechanisms~\ref{regcache} 
(hijacking malloc and other system calls). 
The virtualization needs are driving new and improved IOMMU functionalities 
that would make memory registration obsolete on compliant hardware. 
Unfortunately, it is not widely available at this time.

With portability as an important design goal, the CCI RMA semantic is 
designed to be efficiently implemented on top of the Active Messages 
for devices that do not provide zero-copy support in hardware. 
Open-MX\ref{openmx} has demonstrated high bandwidth and low CPU 
overhead when using the Intel IOA/T copy engine to move data between receive 
buffers of an Ethernet driver in the kernel and an RMA target are in user 
space.

Memory registration produces an handle which has to be communicated to the 
remote side using Active Messages. RMA operations can optionally include a 
small Active Messages, similar to Verbs' Immediate Data. Contrary to Verbs 
and for performance and portability reasons, CCI does not guarantee delivery 
order between RMA transfers, or even for data of a single RMA operation. 
Actually, the InfiniBand Verbs does not guarantee Last-Byte-Written-Last 
semantic as well, but the main IB vendor did and users now assume it. It is 
very common for Verbs applications to poll on the last byte of a RDMA write 
on the target side to determine when the RDMA has completed. This behavior is 
unsafe if the DMA operations are not guaranteed to be delivered in order as 
it is the case on some systems.

Since RMA order is not guaranteed, a Fence flag can be used to selectively 
enforce order on the target side with all previous RMA operations on the 
connection.

...

\note{thoughts?} Explicit memory registration which is missing from MPI.\note{why is this
good, George?} Simpler semantics and coding compared to Verbs.\note{is this true, Dave?}
