\section{Design Goals}
\label{sec:design}

In setting out to design a new communication's interface, we had several goals 
in mind: portability, simplicity, performance, scalability, and robustness.

\subsection{Portability}
Application and middleware developers do not have the resources to continuously 
port their code on different communication interfaces. 
Selecting a vendor-specific API reduces competition in the market place, thus 
increasing prices and adding the risk of business failure or market disruptions. It 
also slows down the adoption of new and improved technology. 
Similarly, vendors do not have the resources to properly support a large set 
of middleware. The whole ecosystem would clearly benefit from a truly unified 
communication layer. 
BSD Sockets and MPI both provide this high-level of portability. 
For any new communication interface to gain acceptance in the broader 
community, it needs to provide a similar breadth of implementations on 
currently available hardware, by supporting the semantics that are common 
to most vendor APIs.

\subsection{Simplicity}
Simplicity is paramount to the success of a programming interface. Critical 
mass cannot be reached by limiting the targeted audience to a few networking 
experts. However, ease of use involves many elements beyond just expertise. 
Code size is a common, albeit subjective, metric used to compare programming 
interfaces. The rationale is that larger codes are harder to debug and 
maintain. For example, an analysis of the Open MPI version 1.4.3 implementation shows substantial
differences between the seven supported communication APIs (excluding self and 
shared memory). The total lines of code of each Byte Transfer Layer (BTL) is 
listed in Table \ref{tab:btl}. The Verbs BTL is the largest, five times the size 
of the TCP sockets BTL, third largest, and 8 to 13 times larger than the BTLs of 
the vendor interfaces. 

\begin{table}[htbp] \centering
\caption{Lines of Code per BTL}
\label{tab:btl}
\begin{IEEEeqnarraybox}[\IEEEeqnarraystrutmode\IEEEeqnarraystrutsizeadd{2pt}{1pt}]{v/s/v/u/v}
\IEEEeqnarrayrulerow\\ &\mbox{BTL}&&Lines of code&\\
\IEEEeqnarraydblrulerow\\
\IEEEeqnarrayseprow[3pt]\\ &Elan&&1,656&\IEEEeqnarraystrutsize{0pt}{0pt}\\
\IEEEeqnarrayseprow[3pt]\\
\IEEEeqnarrayrulerow\\
\IEEEeqnarrayseprow[3pt]\\ &MX&&2,333&\IEEEeqnarraystrutsize{0pt}{0pt}\\
\IEEEeqnarrayseprow[3pt]\\
\IEEEeqnarrayrulerow\\
\IEEEeqnarrayseprow[3pt]\\ &Portals&&2,469&\IEEEeqnarraystrutsize{0pt}{0pt}\\
\IEEEeqnarrayseprow[3pt]\\
\IEEEeqnarrayrulerow\\
\IEEEeqnarrayseprow[3pt]\\ &GM&&2,779&\IEEEeqnarraystrutsize{0pt}{0pt}\\
\IEEEeqnarrayseprow[3pt]\\
\IEEEeqnarrayrulerow\\
\IEEEeqnarrayseprow[3pt]\\ &Sockets (TCP)&&4,192&\IEEEeqnarraystrutsize{0pt}{0pt}\\
\IEEEeqnarrayseprow[3pt]\\
\IEEEeqnarrayrulerow\\
\IEEEeqnarrayseprow[3pt]\\ &UDAPL&&6,208&\IEEEeqnarraystrutsize{0pt}{0pt}\\
\IEEEeqnarrayseprow[3pt]\\
\IEEEeqnarrayrulerow\\
\IEEEeqnarrayseprow[3pt]\\ &OpenIB (Verbs)&&21,574&\IEEEeqnarraystrutsize{0pt}{0pt}\\
\IEEEeqnarrayseprow[3pt]\\
\IEEEeqnarrayrulerow
\end{IEEEeqnarraybox}
\end{table}

Another indicator of complexity is the number of functions available. Choice 
is good but too much choice is worse. Fortunately, software programmers are 
efficient at reducing overly complex interfaces to a minimum set of useful 
semantics.
For example, MPI specifies over 300 functions but the vast majority of MPI 
applications only use a fraction of them. Similarly, relative simplicity was 
the main drive behind the wide adoption of the BSD Socket interface. 
A communication interface should aspire to find the right balance 
between richness of semantics and ease of use.

\subsection{Performance}
Performance is major drive for innovation in networking, from HPC to 
Cloud Computing. All modern network technologies leverage common techniques 
developed in the last two decades: OS-bypass, zero-copy, one-sided and asynchronous operations.

\emph{OS-bypass} allows direct interaction between the application and 
a virtualized instance of the network hardware, without involving the 
operating system. 
This technique is essential for low latency as it removes the need for 
interrupts in the critical path. Furthermore, a process or a thread blocking 
in the kernel is often scheduled on a different core when awakened. Avoiding the 
operating system can greatly improves NUMA locality. 
To support OS-bypass the network adapter must be able to demultiplex incoming 
packets into corresponding queues in each application. Most of this 
functionality is commonly used in Ethernet adapters that support Receive Side 
Scaling (RSS).

\emph{Zero-copy} reduces CPU overhead and increases bandwidth by eliminating 
memory copies in the critical path. The network adapter fetches or delivers 
data directly into the memory space of the application via Direct Memory Access 
(DMA) operations. To this end, the related memory pages must be pinned so 
that the network adapter can safely access it. 
An important drawback of zero-copy is its synchronous nature. Since there is 
no intermediate copies, the memory on the send side cannot be reused until 
the data has safely been delivered on the receive side (or at least put on 
the wire for unreliable connections).

Zero-copy is often confused with \emph{one-sided operations}, which allow a 
communication to complete without the involvement of the application thread 
on the remote side. All of the required information, mainly the remote address 
of the data to access, is provided on the origin side. One-sided operations 
may or may not be associated with zero-copy, and may use the help of a 
progression thread on the target side. Similarly, zero-copy may be implemented 
with receive matching instead or remote addressing, as it is the case with MX 
and Portals.

\emph{Asynchronous operations} are used to decouple the initiation of a 
communication from its progress and subsequent completion. This allows communication to be 
overlapped with computation. 
More practically, asynchronous operations enable initiation of concurrent 
data movements without blocking the application context.

To deliver the best performance, a communication interface should present
semantics that can efficiently leverage all these techniques as provided by 
modern high-speed networks.

\subsection{Scalability}
Projections for leadership scale systems in HPC include hundreds of thousands of nodes 
and millions of cores~\cite{dongarra:exascale-talk-2010}. In the commercial 
space, Cloud Computing data centers are fast approaching these levels. 
In this context, scalability is an important requirement. The time and space 
overhead of a scalable communication interface should not grow linearly with 
the number of communicating partners. BSD Sockets are inefficient in both 
dimensions, as buffers and file handles are allocated for each new socket. 
Through adaptive socket buffering and use of \f{epoll}, Sockets implementations 
have so far managed to reasonably handle large number of connections. 
MPI is inherently more scalable and it has successfully been deployed on large 
HPC machines. However, it is not clear if MPI in its present form can 
efficiently scale to millions of cores. 
Scalability of the Verbs interface was originally quite poor due to its Queue 
Pair model. MPI implementations used various techniques such as connection on 
demand~\cite{Shipman_infinibandscalability} and dynamic buffer management 
to work around the QPs memory footprint problem. Scalability was further
improved with the addition of Shared Receive Queues 
(SRQ)~\cite{shipman07:_inves_infin}, but distinct QPs are still  
required on the send side~\cite{Shipman:2008:XIS:1431669.1431683}. To
address the Cloud Computing and leadership class HPC requirements, a
communication interface should aim for constant buffer and polling
overhead, independently of the number of nodes in the fabric. 

\subsection{Robustness}
Hardware and software failures occur frequently, often proportional with the size of the system. 
As system sizes continue to increase, ignoring such failures will no longer be an option. 
Most MPI implementations currently abort on failures that an application might otherwise tolerate. 
To address this, there have been several efforts aimed at designing fault-tolerant MPI libraries and adding fault recovery 
to the MPI specification. Thus far these efforts have had limited success. The loose semantic about 
status completions was actually a benefit in making MPI a simpler interface, 
developers would send messages and trust MPI to always deliver them. 
Unfortunately, real-world applications eventually had to implement 
checkpoint/restart functionality to tolerate system faults and it is the 
only practical solution available today on large HPC systems. 
Both Sockets and Verbs fare better than MPI on this issue. They use connections 
to represent the state of communication channels without reliance on a single consistent 
distributed process space ({\sf MPI\_COMM\_WORLD}). Connections provide a simplified 
model for robustness; they contain faults and allow for their recovery by resetting 
the state of the affected communication channels. 
Unfortunately, both Sockets and Verbs associate buffers to a connection, which 
negatively affects scalability. A robust and scalable communication interface should provide 
connection-oriented semantics without per-connection resources. 

Communication reliability is often seen as a way to improve overall robustness. 
For some applications such as Media Content Delivery (IPTV), Financial Trading 
(HFT) or system-health monitoring, the provided reliability may be incompatible 
with their timing requirements. 
Furthermore, the most scalable multicast implementations are unreliable. For these 
reasons, a large share of applications use unreliable connections. 
A communication interface should provide different levels of connection 
reliability, as well as support for multicast. 


% LocalWords:  epoll QPs
