\documentclass[conference]{IEEEtran}

\usepackage{cite}

\usepackage{url}

% correct bad hyphenation here
\hyphenation{op-tical net-works semi-conduc-tor Myri-com}

\begin{document}
%
\title{CCI: Common Communication Interface}
% 
\author{\IEEEauthorblockN{Scott Atchley\IEEEauthorrefmark{1}, 
David Dillow\IEEEauthorrefmark{1}, 
Galen Shipman\IEEEauthorrefmark{1}, 
Patrick Geoffray\IEEEauthorrefmark{2} and 
Jeffrey M.\ Squyres\IEEEauthorrefmark{3}}
\IEEEauthorblockA{\IEEEauthorrefmark{1}Oak Ridge National Laboratory, Oak Ridge, TN}
\IEEEauthorblockA{\IEEEauthorrefmark{2}Myricom, Inc., Arcadia, CA}
\IEEEauthorblockA{\IEEEauthorrefmark{3}Cisco Systems, Inc., San Jose, CA}}

% make the title area
\maketitle

\begin{abstract}
The abstract goes here.

\end{abstract}

% no keywords

% For peerreview papers, this IEEEtran command inserts a page break and
% creates the second title. It will be ignored for other modes.
\IEEEpeerreviewmaketitle

\section{Introduction}
Introduction text goes here.

\section{Current State of the Art}
Over the years, many communication interfaces have come and gone. The few that have
remained and seen wide-spread adoption are BSD sockets\cite{BSD}, the Message Passing
Interface (MPI)\cite{MPI}, and some vendor-specific application programming interfaces
(API).

\subsection{Sockets} The socket interface is the most widely used by far. All major
operating systems provide support for sockets and the Internet and all the services it
provides would not exist without it. The popularity of sockets can be attributed to:

\begin{itemize}
\item Simple API
\item Robustness
\item Asynchronous operations
\end{itemize}

The API provides stream and datagram based modes, connection-oriented and connection-less
modes, and client/server semantics for connection-oriented modes. Based on the transport,
the API can supports multiple delivery modes (unicast, multicast, and/or broadcast). The
API does not provide for collective communication nor does it provide one-sided
operations.

Sockets implementations are mature and well understood. It does not assume or require
special hardware features (nor can it exploit them if they exist).

Both sends and receives are buffered allowing operations to complete quickly (if buffer
space is available for sends or data exists in the buffer for receives). Applications may
also choose to not block if need be. The downside of buffering is more work is required by
the CPU which can result in lower throughput over the network.

\section{Goals for CCI}
Text goes here

\section{The CCI Interface}
Text goes here

\section{Evaluation}
Text goes here

\section{Conclusion}
The conclusion goes here.


% use section* for acknowledgement
\section*{Acknowledgment}

The authors would like to thank...


% trigger a \newpage just before the given reference
% number - used to balance the columns on the last page
% adjust value as needed - may need to be readjusted if
% the document is modified later
%\IEEEtriggeratref{8}
% The "triggered" command can be changed if desired:
%\IEEEtriggercmd{\enlargethispage{-5in}}

% references section

% can use a bibliography generated by BibTeX as a .bbl file
% BibTeX documentation can be easily obtained at:
% http://www.ctan.org/tex-archive/biblio/bibtex/contrib/doc/
% The IEEEtran BibTeX style support page is at:
% http://www.michaelshell.org/tex/ieeetran/bibtex/
%\bibliographystyle{IEEEtran}
% argument is your BibTeX string definitions and bibliography database(s)
%\bibliography{IEEEabrv,../bib/paper}
%
% <OR> manually copy in the resultant .bbl file
% set second argument of \begin to the number of references
% (used to reserve space for the reference number labels box)
\begin{thebibliography}{1}

\bibitem{IEEEhowto:kopka}
H.~Kopka and P.~W. Daly, \emph{A Guide to \LaTeX}, 3rd~ed.\hskip 1em plus
  0.5em minus 0.4em\relax Harlow, England: Addison-Wesley, 1999.

\end{thebibliography}

% that's all folks
\end{document}
