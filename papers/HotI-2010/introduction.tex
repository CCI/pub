\section{Introduction}
\label{sec:introduction}

% JMS s/API/interface/g

In 1981, the Internet Protocol (IP)~\cite{RFC791} began the era of
ubiquitous networking.  The Transmission Control Protocol
(TCP)~\cite{RFC793} was layered on top of IP, freeing application
developers from many network-specific issues such as reliable delivery
and message ordering.  Even though TCP/IP was implemented in a wide
variety of compute and networking hardware, developers still needed a
common application programming interface (API) to create portable software.
%
The BSD Sockets API filled that role, and became the {\em de facto}
networking API for most platforms.  

But even though both the BSD Socket wire protocol and API are still
widely used today, the networking hardware that they are used on has
changed radically.
%
For example, typical half round trip (HRT) TCP ping-pong latencies
were measured in multiple milliseconds on early technologies such as
10BASE5 Ethernet, Token Ring, and Token Bus.
%
Modern networking technologies such as 10 Gigabit Ethernet (10GE), Quad Data Rate
(QDR) InfiniBand (IB), Cray's Gemini, IBM's Torrent, 
and SGI's NumaLink have HRT ping pong latencies as low as 
1$\mu$s.

Current generation networking technologies also provide features such
as remote direct memory access (RDMA), operating system (OS) bypass,
``zero copy'' hardware support, one-sided operations, and asynchronous
operations.
%
While some implementations of the BSD Sockets interface can utilize a
subset of these features, its API does not directly expose any of them
to applications.
%
As a direct result, several next-generation network APIs have evolved,
such as the Virtual Interface Architecture (VIA), OpenFabrics Verbs,
Myrinet Express (MX), and the Portals API -- just to name a few.
%
While these interfaces adequately expose the underlying network's
capabilities, none of them have garnered the support from application
developers that the Sockets interface has.
%
Two key barriers exist to widespread adoption: simplicity (from an
application developer perspective) and portability across a variety of
OS and networking technologies (including performance portability).

Application developers are therefore forced to make substantial
tradeoffs in the selection of a user-level network interface for their
network-based applications.
%
While the use of BSD Sockets guarantees portability across nearly
every type of existing network, the emulation of the Sockets API over
an underlying network-native software API can substantially limit both
performance and scalability.
%
On the other hand, the use of a native networking API may satisfy
performance and scalability requirements, but limit the application's
portability to future platforms.

Additionally, metrics that used to be the sole purview of High
Performance Computing (HPC) -- such as ultra-low HRT latencies and
maximum bandwidth at small message sizes -- are becoming important
to other types of common applications, including: internet search
technologies, cloud computing, financial trading, and rich media
serving.
%
Indeed, as the raw bandwidth of network technologies continues to
increase, compute servers need to be able to utilize the entire
network capacity.
%
A portable network interface API is needed that removes the
inefficiencies of TCP, provides convenient abstractions for
applications, and enables portability across a wide variety of OS,
compute server, and network platforms.

In this paper, we propose a new networking interface: the Common
Communication Infrastructure (CCI).  
%
CCI balances the needs of portability and simplicity while preserving
the performance capabilities of advanced networking technologies. 
%
In designing CCI, we have drawn upon prior research with a variety of
low-level networking interfaces as well as our experience in working
directly with application developers in the use of these APIs.
%
Whenever possible, we adhered to our primary goal of simplicity in
order to foster wide-spread adoption, yet preserving both performance
and portability.

The remainder of this paper is organized as follows:
%
Section~\ref{sec:state} depicts the state-of-the-art of the common
messaging APIs, followed by a section describing our designs goals.
%
In Section~\ref{sec:interface}, we detail the CCI abstraction entities
(e.g., {\em endpoints} and {\em connections}) as well as the API.
%
Finally, we present our current implementation performance over UDP, Portals, and MX interfaces, and then conclude the paper.



% LocalWords:  de GBase eXpress
