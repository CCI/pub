\section{Introduction}
\label{sec:introduction}

With the advent of modern networking and the Transmission Control
Protocol (TCP) / Unreliable Datagram Protocol (UDP) providing
connectivity across a diversity of hardware architectures and system
software stacks, the need for a ubiquitous communication API providing
application portability has been acute. The widely used sockets API
has largely fulfilled this role, supporting a variety of underlying
networking technologies through the TCP/UDP abstraction of the lower
level interface. Since the standardization of TCP~\cite{tcp-rfc-675}
in 1974 and UDP~\cite{udp-rfc-768} in 1980 and the socket API that
supported these protocols, networking technologies have made
substantial strides. Early networking technologies supported by
TCP/UDP such as 10BASE5 Ethernet, Token Ring, and Token Bus
technologies with multi-millisecond latencies, have been replaced by
technologies such as 10GBase-SR and Quad Data Rate (QDR) InfiniBand
(IB), Cray's Gemini, IBM's Torrent, and SGI's NumaLink with latencies
as low as 1$\mu$s.

Current generation networking technologies provide advanced features
such as support for remote direct memory access (RDMA), reliable
multicast, and cache injection capabilities. While the sockets
interface supports most of today's advanced networking technologies,
the stream-based API of sockets obfuscates and often negates the
benefits of these features. In response, a number of next-generation
network APIs have been proposed, such as the Virtual Interface
Architecture (VIA), OpenFabrics Verbs, Myrinet eXpress (MX), and the
Portals API. While adequately exposing the underlying network
interface's capabilities, none of these network APIs has garnered the
support from application developers that the Sockets interface has. Two
key barriers exist to widespread adoption of many of these APIs,
simplicity from an application developer perspective, and portability
(including performance portability) across a variety of networking
technologies.

Application developers are currently forced to make substantial
tradeoffs in the selection of an underlying networking API for their
distributed applications. The use of sockets virtually guarantees
portability across a wide range of operating environments but may
substantially limit performance and scalability. The use of a custom
networking API may satisfy performance and scalability requirements
but exposes the application to limited portability and may implicitly
remove flexibility in future infrastructure procurements. As a growing
number of applications in the area of internet search technologies,
cloud computing, financial trading and rich media require the
scalability and performance once reserved for high-performance
computing applications, the need for an API that addresses both
portability and performance is acute.

In this paper we propose a new networking API known as the Common
Communication Infrastructure (CCI) that balances the needs of
portability and simplicity while preserving the performance
capabilities of current and next-generation networking
technologies. In developing CCI we have drawn upon our prior
experience with a variety of low-level networking APIs as well as our
experience in working directly with application developers in the use
of these APIs. We strive, wherever possible, to adhere to our primary
goal of simplicity in order to foster wide-spread adoption while
maintaining performance and portability. The remainder of this paper
is organized as follow.  Section~\ref{sec:state} depicts the
state-of-the-art of the common messaging APIs, followed by a section
describing our designs goals.  In Section~\ref{sec:interface}, we
details the CCI entities, endpoints and connections, as well as the
API. Finally, we present the current implementation performance over
UDP and Portals low level interfaces, and then conclude the paper.


