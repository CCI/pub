\section{Conclusion}
\label{sec:conclusion}

The need for high-performance, low-latency communication, once
reserved primarily for HPC oriented applications now extends to a wide
spectrum of markets such as cloud computing and web services. 
While these applications share similar
performance requirements as their HPC counterparts, they differ in
their need for an adaptive, elastic, distributed computing
model. These applications have historically used the ubiquitous
Sockets interface for portability, sacrificing performance and support for advanced
networking features. As networking technologies have
continued to advance, the gap between achievable performance (as
demonstrated by HPC communication models such as MPI) and realized
performance using the Sockets API has widened. 

We have proposed CCI, a new networking interface to address this gap. 
Our design goals of portability, simplicity, performance, scalability,
and robustness have been driven by the needs of a broad community of
distributed computing application developers. CCI achieves portability
through the use of a component architecture with a clean separation of
API and underlying low-level network driver support. Similar to
Sockets, simplicity has been achieved through a narrow API with
well-defined semantics. High-performance is achieved through a
low-overhead active message style semantic for small/control messages
and RMA support for bulk data movement and zero-copy
semantics. Our prototype implementation over Portals achieves performance that is
within 200ns of the native implementation. The use of shared
resources and minimized per-peer state affords CCI substantially
improved scalability characteristics over alternative API
implementations. CCI maintains as little as 120 bytes per connection
and shared resources on the order of 8 Megabytes per endpoint. Robustness is
achieved through well-defined semantics for a variety of failure
scenarios, allowing the application to respond appropriately to
catastrophic network and remote end-point failure scenarios.  




% LocalWords:  RMA ns
