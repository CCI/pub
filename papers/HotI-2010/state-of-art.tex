\section{State of the Art}
\label{sec:state}
Over the years, many communication interfaces have come and gone. The few that have
remained and seen wide-spread adoption are BSD sockets\cite{Sechrest:CSD-84-191}, the Message Passing
Interface (MPI)\cite{mpi_forum93:_mpi}, and some vendor-specific application programming interfaces
(API).

\subsection{Sockets} The socket interface is the most widely used by far. All major
operating systems provide support for sockets and the Internet and all the services it
provides would not exist without it. The popularity of sockets can be attributed to:

\begin{itemize}
\item Simple API
\item Robustness
\item Implicit buffering
\end{itemize}

The API provides stream and datagram based modes, connection-oriented and connection-less
modes, and client/server semantics for connection-oriented modes. Based on the transport,
the API can supports multiple delivery modes (unicast, multicast, and/or broadcast). The
API does not provide for collective communication nor does it provide one-sided
operations.

Sockets implementations are mature and well understood. It does not assume or require
special hardware features (nor can it exploit them if they exist).

Both sends and receives are buffered allowing operations to complete immediately (if buffer
space is available for sends or data exists in the buffer for receives). Applications may
also choose to not block if send buffers are full or receive buffers are empty if need be.
The downside of buffering is more work is required by the CPU which can result in lower
throughput over the network.

Sockets, when used with SOCK\_STREAM, inherits the well-known TCP
performance constraints~\cite{Foong03tcpperformance} related to the
scaling window and the bandwidth delay product. 

\subsection{MPI}\label{sec:mpi} In the High Performance Computing (HPC) arena, MPI is the dominant
interface for inter-process communication. Designed for maximum scalability, MPI has a
richer, but much more complicated API than Sockets.

MPI provides point-to-point (i.e. send-recv or two-sided semantics), collective, and
one-sided operations. For point-to-point communication, MPI provides a variety of modes
including blocking and non-blocking, synchronous and asynchronous, as well as \emph{ready}
mode (only send if a matching receive has been posted).

Multiple implementations exist\cite{ompi_04_pvmmpi_overview, Gropp:1996:HPI, Liu:2003:RDMA, intel-mpi, platform-mpi}
which support multiple operating systems and/or interconnects.  Rather than connections,
the API uses the notion of communication groups (i.e.  communicators) which include all
processes (MPI\_COMM\_WORLD) and may be split to include subsets of processes. MPI does
provide a notion of dynamic process management which includes MPI\_Comm\_accept and
MPI\_Comm\_connect, but this is less mature and much less used.

The MPI standard does not define an underlying network protocol and each MPI
implementation has written its own network abstraction layer (NAL). These NALs typically
support sockets as well as one or more vendor specific APIs.

MPI is a less mature technology compared to sockets likely due to its
relative complexity (over 100 functions), niche market penetration
(HPC), and relative youth. While MPI semantics are fairly well defined
by the MPI standard, implementations vary in their compliance to the
MPI specification and may exhibit differing behavior when the standard is
silent on a specific semantic. Applications that rely upon the
semantic embodied in a particular implementation may not be portable
across other MPI implementations. MPI provides a rigid fault model in
which a process fault within a communication group imposes failure to
all processes within that communication group
(MPI\_ERRORS\_ABORT). Although work has been done on fault-tolerant
MPI\cite{fagg04:_fault_toler_commun_librar_applic_high_perof, mpi-ft},
that work has yet to be adopted by the broader HPC community. 

The MPI standard remains silent on a number of areas that are often
performance critical. For basic send/recv operations MPI
implementations often adopt two common strategies that attempt to
balance buffering and communication overhead. ``Eager'' mode will send
data immediately to the receiver regardless of the receiver having
posted a matching receive. The data is buffered on the receiver if the
matching receive is not posted upon receipt of the
data. ``Rendezvous'' mode defers sending the data until a posted
receive is matched on the receiver. Performance of an MPI application
can therefore be largely implementation-dependent and may vary even
within a single implementation depending on the MPI configuration
settings used for a particular invocation. Perhaps more importantly,
this ambiguity can result in receiver buffer overruns or out-of-memory
(OOM) faults on the receiver.

MPI has added support for one-sided operations as currently defined
in the MPI\_2 standard~\cite{geist96:_mpi2_lyon}. The MPI\_2 one-sided
interface has received limited adoption due to API complexity and
overly constrained semantics~\cite{bonachea-upc-mpi2}. Current efforts
in the MPI\_3 standardization process are aimed at addressing these
issues. 


\subsection{Vendor APIs} There are numerous vendor- and
organization-specific APIs including OpenFabrics
Verbs\cite{ofa-verbs}, Cray/Sandia's Portals\cite{portals}, Qlogic's
PSM\cite{psm}, Myricom's MX\cite{mx}, LBL's GASNet\cite{gasnet},
LAPI\cite{lapi_a_1998}, DCMF\cite{Kumar:2008:DCM:1375527.1375544} and
so on.  Overall, they provide a lot of choice, but none is perfect
(i.e. none have convinced they others to adopt their API).  Since most
are targeted to specific hardware, the APIs tend to be more
complicated.

\note{Mention DAPL}.
\note{reference last year's RoCee paper}
Here is the cite for the RoCEE paper\cite{RoCEE}.

Based on the earlier VIA specification\cite{via}, the InfiniBand standard does not specify
an API; it only specifies which \emph{verbs} must be supported. After many vendors created
separate Verbs APIs, they eventually coalesced into the OpenFabrics Association's (OFA)
Verbs. OFA's Verbs has had the broadest adoption, yet it still represents less than 43\%
of the machines on the Top 500\cite{top500} with some, but much less, adoption outside of
HPC.

Verbs has support for (small) two-sided and one-sided operations. All operations are
asynchronous. Verbs has support for reliable and unreliable modes, connection-oriented
and connection-less. Verbs does not support buffering; all receives must be posted before
sending. Also, all data movement operations require pinning the memory in advance. For the
most common reliable, connection-oriented communication, Verbs requires that the two
processes establish a queue-pair (QP), which is much more cumbersome than socket's
\f{connect} and \f{accept}. Lastly, requiring QPs between all processes has memory scaling
issues. Work has been done with shared receive queues (SRQ) in MPI\cite{srq} to help
increase the scalability of Verb's based systems.\note{can someone review all of this?}

The Portals API provides one-sided semantics (i.e.  Put/Get), uses match tags to steer
messages to the correct buffers. The API is connection-less and leaves it up to the NAL to
maintain any necessary connection state internally. Portals is mostly used on the large
Cray systems such as ORNL's Jaguar\cite{jaguar_cug_2010} and LANL's Cielo\cite{cielo}.  The
Lustre distributed file system NAL, LNET, was originally based on Portals\cite{lnet}.

Both designed for implementing MPI, Myricom's MyrinetExpress (MX) and Qlogic's PathScale
Messages (PSM) have many similarities. Both provide a two-sided interface which uses
buffering for smaller messages and remote direct memory access for larger messages.  Both
are connection-less in that the target does not have to accept.  Both provide reliable,
in-order matching, but out-of-order completion.

LAPI\cite{lapi_a_1998} and DCMF\cite{Kumar:2008:DCM:1375527.1375544}
are both proprietary software stack developed by IBM, using as target
the RS-series and lately the BG P/Q IBM machines. While some support
from the scientific community outside IBM existed, it failed to
broaden and remains limited to few places. DCMF do support an
interface for one-sided and two-sided message semantic, with
contiguous or not memory layout, dealing with multiple [identical]
interconnects seamlessly. The DCMF communication layer supports in
addition to MPI, multiple programming paradigms such as Aggregate
Remote Memory Copy Interface (ARMCI) and Global Arrays (GA). It
provides in addition to the point-to-point interface, a collective
API, allowing processes to execute asynchronous collective
communications in an optimized way.
